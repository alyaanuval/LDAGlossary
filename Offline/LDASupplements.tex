\documentclass[]{book}
\usepackage{lmodern}
\usepackage{amssymb,amsmath}
\usepackage{ifxetex,ifluatex}
\usepackage{fixltx2e} % provides \textsubscript
\ifnum 0\ifxetex 1\fi\ifluatex 1\fi=0 % if pdftex
  \usepackage[T1]{fontenc}
  \usepackage[utf8]{inputenc}
\else % if luatex or xelatex
  \ifxetex
    \usepackage{mathspec}
  \else
    \usepackage{fontspec}
  \fi
  \defaultfontfeatures{Ligatures=TeX,Scale=MatchLowercase}
\fi
% use upquote if available, for straight quotes in verbatim environments
\IfFileExists{upquote.sty}{\usepackage{upquote}}{}
% use microtype if available
\IfFileExists{microtype.sty}{%
\usepackage{microtype}
\UseMicrotypeSet[protrusion]{basicmath} % disable protrusion for tt fonts
}{}
\usepackage[margin=1in]{geometry}
\usepackage{hyperref}
\hypersetup{unicode=true,
            pdftitle={Supplements for Loss Data Analytics},
            pdfauthor={An open text authored by the Actuarial Community},
            pdfborder={0 0 0},
            breaklinks=true}
\urlstyle{same}  % don't use monospace font for urls
\usepackage{natbib}
\bibliographystyle{apalike}
\usepackage{longtable,booktabs}
\usepackage{graphicx,grffile}
\makeatletter
\def\maxwidth{\ifdim\Gin@nat@width>\linewidth\linewidth\else\Gin@nat@width\fi}
\def\maxheight{\ifdim\Gin@nat@height>\textheight\textheight\else\Gin@nat@height\fi}
\makeatother
% Scale images if necessary, so that they will not overflow the page
% margins by default, and it is still possible to overwrite the defaults
% using explicit options in \includegraphics[width, height, ...]{}
\setkeys{Gin}{width=\maxwidth,height=\maxheight,keepaspectratio}
\IfFileExists{parskip.sty}{%
\usepackage{parskip}
}{% else
\setlength{\parindent}{0pt}
\setlength{\parskip}{6pt plus 2pt minus 1pt}
}
\setlength{\emergencystretch}{3em}  % prevent overfull lines
\providecommand{\tightlist}{%
  \setlength{\itemsep}{0pt}\setlength{\parskip}{0pt}}
\setcounter{secnumdepth}{5}
% Redefines (sub)paragraphs to behave more like sections
\ifx\paragraph\undefined\else
\let\oldparagraph\paragraph
\renewcommand{\paragraph}[1]{\oldparagraph{#1}\mbox{}}
\fi
\ifx\subparagraph\undefined\else
\let\oldsubparagraph\subparagraph
\renewcommand{\subparagraph}[1]{\oldsubparagraph{#1}\mbox{}}
\fi

%%% Use protect on footnotes to avoid problems with footnotes in titles
\let\rmarkdownfootnote\footnote%
\def\footnote{\protect\rmarkdownfootnote}

%%% Change title format to be more compact
\usepackage{titling}

% Create subtitle command for use in maketitle
\newcommand{\subtitle}[1]{
  \posttitle{
    \begin{center}\large#1\end{center}
    }
}

\setlength{\droptitle}{-2em}
  \title{Supplements for Loss Data Analytics}
  \pretitle{\vspace{\droptitle}\centering\huge}
  \posttitle{\par}
  \author{An open text authored by the Actuarial Community}
  \preauthor{\centering\large\emph}
  \postauthor{\par}
  \predate{\centering\large\emph}
  \postdate{\par}
  \date{2018-11-15}

\usepackage{booktabs}

\usepackage{amsthm}
\newtheorem{theorem}{Theorem}[chapter]
\newtheorem{lemma}{Lemma}[chapter]
\theoremstyle{definition}
\newtheorem{definition}{Definition}[chapter]
\newtheorem{corollary}{Corollary}[chapter]
\newtheorem{proposition}{Proposition}[chapter]
\theoremstyle{definition}
\newtheorem{example}{Example}[chapter]
\theoremstyle{definition}
\newtheorem{exercise}{Exercise}[chapter]
\theoremstyle{remark}
\newtheorem*{remark}{Remark}
\newtheorem*{solution}{Solution}
\begin{document}
\maketitle

{
\setcounter{tocdepth}{1}
\tableofcontents
}
\chapter{Purpose}\label{purpose}

\chapter{Glossary}\label{glossary}

\emph{Chapter Preview}.

\section{Making Changes to Terms and
Definitions}\label{making-changes-to-terms-and-definitions}

\begin{itemize}
\tightlist
\item
  First, open up the issues tab on our repository on GitHub
  \href{https://github.com/alyaanuval/LDAGlossary/issues}{here}.
\item
  Click on ``create an issue''.
\item
  Indicate which chapters you want to make changes to in the title.
\item
  Specify the terms and definitions you wish to change, add or remove.
\item
  Click ``Submit new issue''.
\end{itemize}

\section{Terms and Descriptions by
Chapter}\label{terms-and-descriptions-by-chapter}

Detailed purpose.

\subsection{Chapter 1 Introduction to Loss Data
Analytics}\label{chapter-1-introduction-to-loss-data-analytics}

\begin{longtable}[]{@{}cc@{}}
\toprule
\begin{minipage}[b]{0.41\columnwidth}\centering\strut
Term\strut
\end{minipage} & \begin{minipage}[b]{0.42\columnwidth}\centering\strut
Description\strut
\end{minipage}\tabularnewline
\midrule
\endhead
\begin{minipage}[t]{0.41\columnwidth}\centering\strut
analytics\strut
\end{minipage} & \begin{minipage}[t]{0.42\columnwidth}\centering\strut
The process of using data to make decisions. This process involves
gathering data, understanding models of uncertainty,making general
inferences, and communicating results\strut
\end{minipage}\tabularnewline
\begin{minipage}[t]{0.41\columnwidth}\centering\strut
business intelligence\strut
\end{minipage} & \begin{minipage}[t]{0.42\columnwidth}\centering\strut
May focus on processes of collecting data, often through databases and
data warehouses\strut
\end{minipage}\tabularnewline
\begin{minipage}[t]{0.41\columnwidth}\centering\strut
business analytics\strut
\end{minipage} & \begin{minipage}[t]{0.42\columnwidth}\centering\strut
Utilizes tools and methods for statistical analyses of data\strut
\end{minipage}\tabularnewline
\begin{minipage}[t]{0.41\columnwidth}\centering\strut
data science\strut
\end{minipage} & \begin{minipage}[t]{0.42\columnwidth}\centering\strut
Can encompass broader applications in many scientific domains\strut
\end{minipage}\tabularnewline
\begin{minipage}[t]{0.41\columnwidth}\centering\strut
short-term\strut
\end{minipage} & \begin{minipage}[t]{0.42\columnwidth}\centering\strut
Contracts where the insurance coverage is typically provided for six
months or a year\strut
\end{minipage}\tabularnewline
\begin{minipage}[t]{0.41\columnwidth}\centering\strut
property insurance\strut
\end{minipage} & \begin{minipage}[t]{0.42\columnwidth}\centering\strut
In the US, policies such as renters and homeowners\strut
\end{minipage}\tabularnewline
\begin{minipage}[t]{0.41\columnwidth}\centering\strut
casualty insurance\strut
\end{minipage} & \begin{minipage}[t]{0.42\columnwidth}\centering\strut
In the US, a policy such as auto that covers medical damages to
people\strut
\end{minipage}\tabularnewline
\begin{minipage}[t]{0.41\columnwidth}\centering\strut
nonlife or general insurance\strut
\end{minipage} & \begin{minipage}[t]{0.42\columnwidth}\centering\strut
In the rest of the world, property and casualty insurance are both known
as nonlife or general insurance, to distinguish them from life
insurance\strut
\end{minipage}\tabularnewline
\begin{minipage}[t]{0.41\columnwidth}\centering\strut
underwriting\strut
\end{minipage} & \begin{minipage}[t]{0.42\columnwidth}\centering\strut
The process of classifying risks into homogenous categoriesand assigning
policyholders to these categories, lies at the core of ratemaking.
Policyholders within a class have similar risk profiles and so are
charged thesame insurance price\strut
\end{minipage}\tabularnewline
\begin{minipage}[t]{0.41\columnwidth}\centering\strut
ratemaking\strut
\end{minipage} & \begin{minipage}[t]{0.42\columnwidth}\centering\strut
Where analysts seek to determine the right price for the right
risk\strut
\end{minipage}\tabularnewline
\begin{minipage}[t]{0.41\columnwidth}\centering\strut
experience rating or merit rating\strut
\end{minipage} & \begin{minipage}[t]{0.42\columnwidth}\centering\strut
Modifying premiums with claims history\strut
\end{minipage}\tabularnewline
\begin{minipage}[t]{0.41\columnwidth}\centering\strut
claims adjustment\strut
\end{minipage} & \begin{minipage}[t]{0.42\columnwidth}\centering\strut
The process of determining coverage, legal liability, and settling
claims\strut
\end{minipage}\tabularnewline
\begin{minipage}[t]{0.41\columnwidth}\centering\strut
claims leakage\strut
\end{minipage} & \begin{minipage}[t]{0.42\columnwidth}\centering\strut
Dollars lost through claims management inefficiencies\strut
\end{minipage}\tabularnewline
\begin{minipage}[t]{0.41\columnwidth}\centering\strut
loss reserving\strut
\end{minipage} & \begin{minipage}[t]{0.42\columnwidth}\centering\strut
Setting aside money for unpaid claims\strut
\end{minipage}\tabularnewline
\begin{minipage}[t]{0.41\columnwidth}\centering\strut
claim\strut
\end{minipage} & \begin{minipage}[t]{0.42\columnwidth}\centering\strut
At a fundamental level, insurance companies accept premiums in exchange
for promises to indemnify a policyholder upon the uncertain occurrence
of an insured event. This indemnification is known as aÂ~claim\strut
\end{minipage}\tabularnewline
\begin{minipage}[t]{0.41\columnwidth}\centering\strut
severity\strut
\end{minipage} & \begin{minipage}[t]{0.42\columnwidth}\centering\strut
A positive amount is a key financial expenditure for an insurer. So,
knowing only the claim amount summarizes the reimbursement to the
policyholder\strut
\end{minipage}\tabularnewline
\begin{minipage}[t]{0.41\columnwidth}\centering\strut
frequency\strut
\end{minipage} & \begin{minipage}[t]{0.42\columnwidth}\centering\strut
How often claims arise\strut
\end{minipage}\tabularnewline
\begin{minipage}[t]{0.41\columnwidth}\centering\strut
pure premium or loss cost\strut
\end{minipage} & \begin{minipage}[t]{0.42\columnwidth}\centering\strut
The total severity divided by the number of claims\strut
\end{minipage}\tabularnewline
\begin{minipage}[t]{0.41\columnwidth}\centering\strut
rating variables\strut
\end{minipage} & \begin{minipage}[t]{0.42\columnwidth}\centering\strut
Externally available variables useful in setting insurance rates and
premiums\strut
\end{minipage}\tabularnewline
\bottomrule
\end{longtable}

\subsection{Chapter 2 Frequency
Modeling}\label{chapter-2-frequency-modeling}

\begin{longtable}[]{@{}cc@{}}
\toprule
\begin{minipage}[b]{0.43\columnwidth}\centering\strut
Term\strut
\end{minipage} & \begin{minipage}[b]{0.43\columnwidth}\centering\strut
Description\strut
\end{minipage}\tabularnewline
\midrule
\endhead
\begin{minipage}[t]{0.43\columnwidth}\centering\strut
claim\strut
\end{minipage} & \begin{minipage}[t]{0.43\columnwidth}\centering\strut
compensation from insurer to insured upon the occurrence of an insured
event\strut
\end{minipage}\tabularnewline
\begin{minipage}[t]{0.43\columnwidth}\centering\strut
frequency\strut
\end{minipage} & \begin{minipage}[t]{0.43\columnwidth}\centering\strut
how often claims arise or how often insured event occurs\strut
\end{minipage}\tabularnewline
\begin{minipage}[t]{0.43\columnwidth}\centering\strut
severity\strut
\end{minipage} & \begin{minipage}[t]{0.43\columnwidth}\centering\strut
amount of each payment for an insured event\strut
\end{minipage}\tabularnewline
\begin{minipage}[t]{0.43\columnwidth}\centering\strut
expected cost\strut
\end{minipage} & \begin{minipage}[t]{0.43\columnwidth}\centering\strut
expected number of claims (frequency) times expected amount per claim
(severity)\strut
\end{minipage}\tabularnewline
\begin{minipage}[t]{0.43\columnwidth}\centering\strut
binomial distribution\strut
\end{minipage} & \begin{minipage}[t]{0.43\columnwidth}\centering\strut
discrete frequency distribution and member of (a, b, 0) class; for
number of successes in a fixed number of independent identical trials
with binary outcomes\strut
\end{minipage}\tabularnewline
\begin{minipage}[t]{0.43\columnwidth}\centering\strut
negative binomial distribution\strut
\end{minipage} & \begin{minipage}[t]{0.43\columnwidth}\centering\strut
discrete frequency distribution and member of (a, b, 0) class; for
number of successes until we observe the r-th failure in independent
identical trials with binary outcomes\strut
\end{minipage}\tabularnewline
\begin{minipage}[t]{0.43\columnwidth}\centering\strut
poisson distribution\strut
\end{minipage} & \begin{minipage}[t]{0.43\columnwidth}\centering\strut
discrete frequency distribution and member of (a, b, 0) class; for
independent events occuring at a constant rate in a certain time
period\strut
\end{minipage}\tabularnewline
\begin{minipage}[t]{0.43\columnwidth}\centering\strut
likelihood\strut
\end{minipage} & \begin{minipage}[t]{0.43\columnwidth}\centering\strut
observed value of mass function\strut
\end{minipage}\tabularnewline
\begin{minipage}[t]{0.43\columnwidth}\centering\strut
maximum likelihood estimator (mle)\strut
\end{minipage} & \begin{minipage}[t]{0.43\columnwidth}\centering\strut
to find parameter values that produce the largest likelihood\strut
\end{minipage}\tabularnewline
\begin{minipage}[t]{0.43\columnwidth}\centering\strut
risk\strut
\end{minipage} & \begin{minipage}[t]{0.43\columnwidth}\centering\strut
a unit covered by insurance\strut
\end{minipage}\tabularnewline
\begin{minipage}[t]{0.43\columnwidth}\centering\strut
parameter\strut
\end{minipage} & \begin{minipage}[t]{0.43\columnwidth}\centering\strut
a numerical characteristic of a population\strut
\end{minipage}\tabularnewline
\begin{minipage}[t]{0.43\columnwidth}\centering\strut
mixture\strut
\end{minipage} & \begin{minipage}[t]{0.43\columnwidth}\centering\strut
mixture of subgroups, each with their own distribution\strut
\end{minipage}\tabularnewline
\begin{minipage}[t]{0.43\columnwidth}\centering\strut
fitted distribution\strut
\end{minipage} & \begin{minipage}[t]{0.43\columnwidth}\centering\strut
distribution used for modeling the data\strut
\end{minipage}\tabularnewline
\begin{minipage}[t]{0.43\columnwidth}\centering\strut
Pearson chi-square statistic\strut
\end{minipage} & \begin{minipage}[t]{0.43\columnwidth}\centering\strut
to check for the goodness of fit of the fitted distribution\strut
\end{minipage}\tabularnewline
\bottomrule
\end{longtable}

\section{Terms and Chapter First
Defined}\label{terms-and-chapter-first-defined}

Detailed purpose.

\begin{longtable}[]{@{}cc@{}}
\toprule
\begin{minipage}[b]{0.43\columnwidth}\centering\strut
Term\strut
\end{minipage} & \begin{minipage}[b]{0.30\columnwidth}\centering\strut
Chapter first defined\strut
\end{minipage}\tabularnewline
\midrule
\endhead
\begin{minipage}[t]{0.43\columnwidth}\centering\strut
analytics\strut
\end{minipage} & \begin{minipage}[t]{0.30\columnwidth}\centering\strut
1\strut
\end{minipage}\tabularnewline
\begin{minipage}[t]{0.43\columnwidth}\centering\strut
binomial distribution\strut
\end{minipage} & \begin{minipage}[t]{0.30\columnwidth}\centering\strut
2\strut
\end{minipage}\tabularnewline
\begin{minipage}[t]{0.43\columnwidth}\centering\strut
business analytics\strut
\end{minipage} & \begin{minipage}[t]{0.30\columnwidth}\centering\strut
1\strut
\end{minipage}\tabularnewline
\begin{minipage}[t]{0.43\columnwidth}\centering\strut
business intelligence\strut
\end{minipage} & \begin{minipage}[t]{0.30\columnwidth}\centering\strut
1\strut
\end{minipage}\tabularnewline
\begin{minipage}[t]{0.43\columnwidth}\centering\strut
casualty insurance\strut
\end{minipage} & \begin{minipage}[t]{0.30\columnwidth}\centering\strut
1\strut
\end{minipage}\tabularnewline
\begin{minipage}[t]{0.43\columnwidth}\centering\strut
claim\strut
\end{minipage} & \begin{minipage}[t]{0.30\columnwidth}\centering\strut
1\strut
\end{minipage}\tabularnewline
\begin{minipage}[t]{0.43\columnwidth}\centering\strut
claims adjustment\strut
\end{minipage} & \begin{minipage}[t]{0.30\columnwidth}\centering\strut
1\strut
\end{minipage}\tabularnewline
\begin{minipage}[t]{0.43\columnwidth}\centering\strut
claims leakage\strut
\end{minipage} & \begin{minipage}[t]{0.30\columnwidth}\centering\strut
1\strut
\end{minipage}\tabularnewline
\begin{minipage}[t]{0.43\columnwidth}\centering\strut
data science\strut
\end{minipage} & \begin{minipage}[t]{0.30\columnwidth}\centering\strut
1\strut
\end{minipage}\tabularnewline
\begin{minipage}[t]{0.43\columnwidth}\centering\strut
expected cost\strut
\end{minipage} & \begin{minipage}[t]{0.30\columnwidth}\centering\strut
2\strut
\end{minipage}\tabularnewline
\begin{minipage}[t]{0.43\columnwidth}\centering\strut
experience rating or merit rating\strut
\end{minipage} & \begin{minipage}[t]{0.30\columnwidth}\centering\strut
1\strut
\end{minipage}\tabularnewline
\begin{minipage}[t]{0.43\columnwidth}\centering\strut
fitted distribution\strut
\end{minipage} & \begin{minipage}[t]{0.30\columnwidth}\centering\strut
NA\strut
\end{minipage}\tabularnewline
\begin{minipage}[t]{0.43\columnwidth}\centering\strut
frequency\strut
\end{minipage} & \begin{minipage}[t]{0.30\columnwidth}\centering\strut
1\strut
\end{minipage}\tabularnewline
\begin{minipage}[t]{0.43\columnwidth}\centering\strut
likelihood\strut
\end{minipage} & \begin{minipage}[t]{0.30\columnwidth}\centering\strut
NA\strut
\end{minipage}\tabularnewline
\begin{minipage}[t]{0.43\columnwidth}\centering\strut
loss reserving\strut
\end{minipage} & \begin{minipage}[t]{0.30\columnwidth}\centering\strut
1\strut
\end{minipage}\tabularnewline
\begin{minipage}[t]{0.43\columnwidth}\centering\strut
maximum likelihood estimator (mle)\strut
\end{minipage} & \begin{minipage}[t]{0.30\columnwidth}\centering\strut
2\strut
\end{minipage}\tabularnewline
\begin{minipage}[t]{0.43\columnwidth}\centering\strut
mixture\strut
\end{minipage} & \begin{minipage}[t]{0.30\columnwidth}\centering\strut
NA\strut
\end{minipage}\tabularnewline
\begin{minipage}[t]{0.43\columnwidth}\centering\strut
negative binomial distribution\strut
\end{minipage} & \begin{minipage}[t]{0.30\columnwidth}\centering\strut
NA\strut
\end{minipage}\tabularnewline
\begin{minipage}[t]{0.43\columnwidth}\centering\strut
nonlife or general insurance\strut
\end{minipage} & \begin{minipage}[t]{0.30\columnwidth}\centering\strut
1\strut
\end{minipage}\tabularnewline
\begin{minipage}[t]{0.43\columnwidth}\centering\strut
parameter\strut
\end{minipage} & \begin{minipage}[t]{0.30\columnwidth}\centering\strut
NA\strut
\end{minipage}\tabularnewline
\begin{minipage}[t]{0.43\columnwidth}\centering\strut
Pearson chi-square statistic\strut
\end{minipage} & \begin{minipage}[t]{0.30\columnwidth}\centering\strut
NA\strut
\end{minipage}\tabularnewline
\begin{minipage}[t]{0.43\columnwidth}\centering\strut
poisson distribution\strut
\end{minipage} & \begin{minipage}[t]{0.30\columnwidth}\centering\strut
2\strut
\end{minipage}\tabularnewline
\begin{minipage}[t]{0.43\columnwidth}\centering\strut
property insurance\strut
\end{minipage} & \begin{minipage}[t]{0.30\columnwidth}\centering\strut
1\strut
\end{minipage}\tabularnewline
\begin{minipage}[t]{0.43\columnwidth}\centering\strut
pure premium or loss cost\strut
\end{minipage} & \begin{minipage}[t]{0.30\columnwidth}\centering\strut
1\strut
\end{minipage}\tabularnewline
\begin{minipage}[t]{0.43\columnwidth}\centering\strut
ratemaking\strut
\end{minipage} & \begin{minipage}[t]{0.30\columnwidth}\centering\strut
1\strut
\end{minipage}\tabularnewline
\begin{minipage}[t]{0.43\columnwidth}\centering\strut
rating variables\strut
\end{minipage} & \begin{minipage}[t]{0.30\columnwidth}\centering\strut
1\strut
\end{minipage}\tabularnewline
\begin{minipage}[t]{0.43\columnwidth}\centering\strut
risk\strut
\end{minipage} & \begin{minipage}[t]{0.30\columnwidth}\centering\strut
NA\strut
\end{minipage}\tabularnewline
\begin{minipage}[t]{0.43\columnwidth}\centering\strut
severity\strut
\end{minipage} & \begin{minipage}[t]{0.30\columnwidth}\centering\strut
1\strut
\end{minipage}\tabularnewline
\begin{minipage}[t]{0.43\columnwidth}\centering\strut
short-term\strut
\end{minipage} & \begin{minipage}[t]{0.30\columnwidth}\centering\strut
1\strut
\end{minipage}\tabularnewline
\begin{minipage}[t]{0.43\columnwidth}\centering\strut
underwriting\strut
\end{minipage} & \begin{minipage}[t]{0.30\columnwidth}\centering\strut
1\strut
\end{minipage}\tabularnewline
\bottomrule
\end{longtable}

\chapter{Table of Distributions}\label{table-of-distributions}

Detailed Purpose.

\[
{\small
\begin{matrix}
\begin{array}{|l|cccccc|}
\hline
      \text{Name} & \text{Probability Density}  &       \text{Mean} & \text{Variance } \sigma^2 & \text{Moments } \mu_k'=\mathrm{E~}X^k & \mathrm{E~}(X\wedge x)^k & \text{Moment Generating} \\
           & \text{Function f(x)}  & \mu=\mathrm{E~}X & \mathrm{E~}(X-\mu)^2 & \text{or } \mu_k=\mathrm{E~}(X-\mu)^k &            & \text{Function M(t)}=\mathrm{E~}e^{tX} \\
\hline
   \text{Uniform} & \frac{1}{\beta-\alpha} & \frac{\beta+\alpha}{2} & \frac{(\beta-\alpha)^2}{12} & \mu_k=0 \text{ for k odd} &            & \frac{e^{\beta t}-e^{\alpha t}}{(\beta-\alpha)t} \\
           & -\infty<\alpha, <\beta<\infty &  & & \mu_k=\frac{(\beta-\alpha)^k}{2^k(k+1)} \text{ for k even} &            & \\
\hline
    \text{Normal} & \frac{1}{\sqrt{2\pi}\sigma}\exp\left( -\frac{(x-\mu)^2}{2\sigma^2}\right)  &     \mu & \sigma^2 & \mu_k=0 \text{ for k odd} &            & \exp(\mu t+\sigma^2~t^2/2) \\
           & -\infty<\mu<\infty, \sigma>0  &            &            & \mu_k=\frac{k!\sigma^k}{(k/2)!2^{k/2}} \text{ for k even} &            &            \\
\hline
\text{Exponential} & \frac{1}{\theta}e^{-x/\theta} &  \theta & \theta^2 & \mu_k'=\theta^k \Gamma (k+1) & \theta^k\Gamma (k+1)\Gamma (k+1;x/\theta) & \frac{1}{1-\theta~t} \\
           & \lambda>0 &            &            &            & +x^k e^{-x/\theta} & \\
\hline
     \text{Gamma} & \frac{1}{\theta^{\alpha}\Gamma(\alpha)}x^{\alpha-1}e^{-x/\theta} & \alpha~\theta & \alpha~\theta^2 & \mu_k'=\frac{\theta^k\Gamma(k+\alpha)}{\Gamma(\alpha)} & \frac{\theta^k\Gamma(k+\alpha)}{\Gamma(\alpha)}\Gamma(k+\alpha; x/\theta) & \frac{1}{(1-\theta t)^{\alpha}} \\
           & \theta>0, \alpha>0 &            &            & & +x^k[1-\Gamma(\alpha; x/\theta)] & \\
\hline
      \text{Beta} & \frac{1}{B(a,b)} u^a(1-u)^{b-1}\frac{\theta}{x}, & \frac{a \theta}{a+b} & \frac{ab\theta^2}{(a+b+1)(a+b)^2} & \mu_k'=\theta^k \frac{B(k+a,b)}{B(a,b)} & \text{Not useful} & \text{Not useful} \\
           & u=x/\theta ,  a>0 , b>0 & & & &            &            \\
\hline
    \text{Cauchy} & \frac{1}{\pi\beta}[1+\left( \frac{x-\alpha}{\beta}\right)^2]^{-1} &   \text{Does not} & \text{Does not exist} & \text{Does not exist} & \text{Does not exist} & \text{Does not exist} \\
           & -\infty <\alpha <\infty, \beta>0 &      \text{exist} &            &            &            &            \\
\hline
 \text{Lognormal} &    & \exp(\mu+ & \exp(2\mu +2\sigma^2)- & \mu_k'=\exp(k\mu+k\sigma^2) & \exp(k\mu+k\sigma^2) & \text{Not useful} \\
           & \frac{1}{x\sqrt{2\pi}\sigma}\exp\left( -\frac{(\ln x-\mu)^2}{2\sigma^2}\right)  & \sigma^2/2) & \exp(2\mu+\sigma^2) & & \Phi\left( \frac{lnx-\mu-k\sigma^2}{\sigma}\right) &            \\
           & -\infty <\mu <\infty, \sigma>0 &            &            &            & +x^k(1-F(x)) &            \\
\hline
    \text{Pareto} & \frac{\alpha \theta^{\alpha}}{x^{\alpha+1}}, \alpha>0 & \frac{\alpha\theta}{\alpha-1} & \frac{\alpha\theta^2}{\alpha-2}-\left( \frac{\alpha \theta}{\alpha-1}\right)^2 & \mu_k'=\frac{\alpha\theta^k}{\alpha-k} & \frac{\alpha\theta^k}{\alpha-k}-\frac{k\theta^{\alpha}}{(\alpha-k)x^{\alpha-k}} & \text{Does not exist} \\
\hline
\end{array}
\end{matrix}
}
\]

\chapter{Conventions for Notation}\label{S:NotationConvention}

\emph{Chapter Preview}. \textbf{Loss Data Analytics} will serve as a
bridge between actuarial problems and methods and widely accepted
statistical concepts and tools. Thus, the notation should be consistent
with standard usage employed in probability and mathematical statistics.
See, for example, \citep{halperin1965recommended} for a description of
one standard.

\section{General Conventions}\label{S:General}

\begin{itemize}
\tightlist
\item
  Random variables are denoted by upper-case italicized Roman letters,
  with \(X\) or \(Y\) denoting a claim size variable, \(N\) a claim
  count variable, and \(S\) an aggregate loss variable. Realizations of
  random variables are denoted by corresponding lower-case italicized
  Roman letters, with \(x\) or \(y\) for claim sizes, \(n\) for a claim
  count, and \(s\) for an aggregate loss.
\item
  Probability events are denoted by upper-case Roman letters, such as
  \(\Pr(\mathrm{A})\) for the probability that an outcome in the event
  `'A'' occurs.
\item
  Cumulative probability functions are denoted by \(F(z)\) and
  probability density functions by the associated lower-case Roman
  letter: \(f(z)\).
\item
  For distributions, parameters are denoted by lower-case Greek letters.
  A caret or `'hat'' indicates a sample estimate of the corresponding
  population parameter. For example, \(\hat{\beta}\) is an estimate of
  \(\beta\) .
\item
  The arithmetic mean of a set of numbers, say, \(x_1, \ldots, x_n\), is
  usually denoted by \(\bar{x}\); the use of \(x\), of course, is
  optional.
\item
  Use upper-case boldface Roman letters to denote a matrix other than a
  vector. Use lower-case boldface Roman letters to denote a (column)
  vector. Use a superscript prime `'\(\prime\)'' for transpose. For
  example, \(\mathbf{x}^{\prime} \mathbf{A} \mathbf{x}\) is a quadratic
  form.
\item
  Acronyms are to be used sparingly, given the international focus of
  our audience. Introduce acronyms commonly used in statistical
  nomenclature but limit the number of acronyms introduced. For example,
  \emph{pdf} for probability density function is useful but \emph{GS}
  for Gini statistic is not.
\end{itemize}

\section{Abbreviations}\label{S:Abbreviations}

Here is a list of abbreviations that we adopt. We italicize these
acronyms. For example, we can discuss the goodness of fit in terms of
the \emph{AIC} criterion.

\[
\begin{array}{ll}
\hline
AIC & \text{Akaike information criterion} \\
BIC & \text{(Schwarz) Bayesian information criterion} \\
cdf & \text{cumulative distribution function} \\
df & \text{degrees of freedom} \\
iid & \text{independent and identically distributed} \\
glm & \text{generalized linear model} \\
mle & \text{maximum likelihood estimate}\\
ols & \text{ordinary least squares} \\
pdf & \text{probability density function} \\
pf  & \text{probability  function} \\
pmf & \text{probability mass function} \\
rv & \text{random variable} \\ \hline
\end{array}
\]

\section{Common Statistical Symbols and Operators}\label{S:StatSymbols}

Here is a list of commonly used statistical symbols and operators,
including the latex code that we use to generate them (in the parens).

\[
\begin{array}{cl}  \hline
I(\cdot) & \text{binary indicator operator (}I\text{). For example, }I(A) \text{ is one if an outcome in event} \\
& \ \ \ \ \  A \text{ occurs and is 0 otherwise.} \\
\Pr(\cdot) & \text{probability }(\backslash{\tt{Pr}}) \\
\mathrm{E}(\cdot)  & {\text{expectation operator }} (\backslash{\tt{mathrm\{E\}}}). {\text{ For example, }} \mathrm{E}(X)=\mathrm{E}~X {\text{ is the }} \\
& \ \ \ \ \ {\text{expected value of the random variable }}X,{\text{ commonly denoted by }}\mu. \\
\mathrm{Var}(\cdot)  & \text{variance operator }(\backslash{\tt{mathrm\{Var\}}}). \text{For example, } \mathrm{Var}(X)=\mathrm{Var}~X\text{ is the} \\
& \ \ \ \ \  \text{ variance of the random variable } X, \text{commonly denoted by } \sigma^2. \\
\mu_k = \mathrm{E}~X^k & \text{kth moment of the random variable X. For k=1, use }\mu = \mu_1. \\
\mathrm{Cov}(\cdot,\cdot)  & \text{covariance operator } (\backslash{\tt{mathrm\{Cov\}}}).\text{ For example, } \\
& \ \ \ \ \ \mathrm{Cov}(X,Y)=\mathrm{E}\left\{(X -\mathrm{E}~X)(Y-\mathrm{E}~Y)\right\}  =\mathrm{E}(XY) -(\mathrm{E}~X)(\mathrm{E}~Y)\\
& \ \ \ \ \  \text{ is the covariance between random variables }X\text{ and }Y. \\
\mathrm{E}(X | \cdot)  & \text{conditional expectation operator. For example, }\mathrm{E}(X |Y=y) \text{ is the}\\
& \ \ \ \ \   \text{ conditional expected value of a random variable }X\text{ given that the random variable }Y\text{ equals y. }\\
\Phi(\cdot) & \text{standard normal cumulative distribution function }(\backslash{\tt{Phi}})\\
\phi(\cdot) & \text{standard normal probability density function }(\backslash{\tt{phi}})\\
\sim & \text{means is distributed as }(\backslash{\tt{sim}}). \text{ For example, }X\sim F \text{ means that the } \\
& \ \ \ \ \  \text{random variable } x \text{ has distribution function }F. \\
se(\hat{\beta}) & \text{standard error of the parameter estimate }\hat{\beta} ~ (\backslash{\tt{hat\{}}\backslash{\tt{beta\}}}), \text{ usually }\\
& \ \ \ \ \  \text{ an estimate of the standard deviation of }\hat{\beta},\text{ which is }\sqrt{Var(\hat{\beta})}. \\
H_0 &  \text{null hypothesis} \\
H_a \text{ or }H_1 & \text{alternative hypothesis} \\
\hline
\end{array}
\]

\section{Common Mathematical Symbols and Functions}\label{S:Symbols}

Here is a list of commonly used mathematical symbols and functions,
including the latex code that we use to generate them (in the parens).

\[
\begin{array}{cl}
\hline
\equiv & \text{identity, equivalence }(\backslash\tt{equiv}) \\
a:=b   & \text{defines a in terms of }b \\
\implies     & \text{implies }(\backslash\tt{implies})\\
\iff  & \text{if and only if }(\backslash\tt{iff})\\
\to, \longrightarrow & \text{converges to }(\backslash\tt{to}, \backslash\tt{longrightarrow}) \\
\mathbb{N} & \text{natural numbers }1,2,\ldots ( \backslash\tt{mathbb\{N\}}) \\
\mathbb{R} & \text{real numbers }(\backslash\tt{mathbb\{R\}})\\
\in        & \text{belongs to }(\backslash\tt{in}) \\
\notin     & \text{does not belong to }(\backslash\tt{notin}) \\
\subseteq  & \text{is a subset of }(\backslash\tt{subseteq}) \\
\subset    & \text{is a proper subset of }(\backslash\tt{subset}) \\
\cup       & \text{union  }(\backslash\tt{cup}) \\
\cap       & \text{intersection  }(\backslash\tt{cap}) \\
\emptyset  & \text{empty set }(\backslash\tt{emptyset})  \\
A^{c}      & \text{complement of }A   \\
g*f        & \text{convolution }(g*f)(x)=\int_{-\infty}^{\infty}g(y)f(x-y)dy \\
\exp       & \text{exponential }(\backslash\tt{exp}) \\
\log       & \text{natural logarithm }(\backslash\tt{log})\\
\log_a     & \text{logarithm to the base }a \\
!          & \text{factorial} \\
\text{sgn}(x)    & \text{sign of x}(\tt{sgn}) \\
\lfloor x\rfloor & \text{integer part of x, that is, largest integer }\leq x \\
                 & (\backslash\tt{lfloor}, \backslash\tt{rfloor}) \\
|x|        & \text{absolute value of scalar }x \\
\varGamma(x) & \text{gamma (generalized factorial) function } (\backslash\tt{varGamma}),\\
           & \text{satisfying }\varGamma(x+1)=x\varGamma(x) (\tt{\varGamma}) \\
B(x,y)     & \text{beta function, }\varGamma(x)\varGamma(y)/\varGamma(x+y) \\
\hline
\end{array}
\]

\section{Further Readings}\label{further-readings}

To make connections to other literatures, see \citep{abadir2002notation}
\url{http://www.janmagnus.nl/misc/notation.zip} for a summary of
notation from the econometrics perspective. This reference has a
terrific feature that many latex symbols are defined in the article.
Further, there is a long history of discussion and debate surrounding
actuarial notation; see \citep{boehm1975thoughts} for one contribution.

\bibliography{book.bib,packages.bib}


\end{document}
